\documentclass[12pt]{article}

\usepackage{amsmath, amsfonts, amsthm}
\usepackage{hyperref}
\usepackage{graphicx}
\usepackage{geometry}
\geometry{
	a4paper,
 	left=24mm,
 	right=24mm,
 	top=30mm,
 	bottom=35mm
}
\usepackage{color}
\definecolor{bluekeywords}{rgb}{0.13,0.13,1}
\definecolor{greencomments}{rgb}{0,0.5,0}
\definecolor{redstrings}{rgb}{0.9,0,0}
\definecolor{cyantypes}{RGB}{0,183,235}

\usepackage{listings}
\lstdefinelanguage{FSharp}%
{morekeywords=[1]{let, new, match, with, rec, open, module, namespace, type, of, member, % 
and, for, while, true, false, in, do, begin, end, fun, function, return, yield, try, %
mutable, if, then, else, cloud, async, static, use, abstract, interface, inherit, finally },
  morekeywords=[2]{double, float, int, string, List, BigInteger},
  otherkeywords={ let!, return!, do!, yield!, use!, var, select, where, order, by },
% otherkeywords={from}
  keywordstyle=[1]\color{bluekeywords},
  keywordstyle=[2]\color{cyantypes},
  sensitive=true,
  basicstyle=\ttfamily,
	breaklines=true,
  xleftmargin=\parindent,
  aboveskip=\bigskipamount,
	tabsize=4,
  morecomment=[l][\color{greencomments}]{///},
  morecomment=[l][\color{greencomments}]{//},
  morecomment=[s][\color{greencomments}]{{(*}{*)}},
  morestring=[b]",
  showstringspaces=false,
  literate={`}{\`}1,
  stringstyle=\color{redstrings},
}
\definecolor{codegreen}{rgb}{0,0.6,0}
\definecolor{codegray}{rgb}{0.5,0.5,0.5}
\definecolor{codepurple}{rgb}{0.58,0,0.82}
\definecolor{backcolour}{rgb}{0.95,0.95,0.92}
\lstdefinestyle{mystyle}{
    backgroundcolor=\color{backcolour},   
    commentstyle=\color{codegreen},
    keywordstyle=\color{magenta},
    numberstyle=\tiny\color{codegray},
    stringstyle=\color{codepurple},
    basicstyle=\footnotesize\ttfamily,
    breakatwhitespace=false,         
    breaklines=true,                 
    captionpos=b,                    
    keepspaces=true,                 
    numbers=left,                    
    numbersep=5pt,                  
    showspaces=false,                
    showstringspaces=false,
    showtabs=false,                  
    tabsize=4
}
\lstset{style=mystyle}

\newtheorem*{question*}{Question}
\newtheorem*{modQuestion*}{Modified Question}
\newtheorem*{origQuestion*}{Original Question}
\begin{document}


\begin{center}

{\large F\# Tutorial\\} \vspace{2mm}
\textbf{\LARGE Pipe-Forward Operator}\\
\vspace{1.5mm}
{\Large\emph{\today}}

\end{center}


\section{Modified Project Euler Solutions}

\subsubsection*{\texttt{IsPrime} Function Provided}
The following function determines whether a positive integer \texttt{x} is a prime number or not. It is already provided, and we do not need to re-implment it.
\begin{lstlisting}[language=FSharp]
let IsPrime x =
    let squareRoot = x |> double |> sqrt |> int 
    if x = 1 then false
    else if x = 2 then true
    else if x % 2 = 0 then false
    else 
        [3 .. 2 .. squareRoot]
        |> List.forall (fun i -> x%i <> 0)

// val IsPrime: x:int -> bool
\end{lstlisting}

\subsection*{Question 1} 

\url{https://projecteuler.net/problem=1}

\begin{origQuestion*}
Implement a function that sums up all multiples of $3$ or $5$ in a list.
\end{origQuestion*}

\begin{lstlisting}[language=FSharp]
let SumMultiplesOf3Or5 xList =
    xList
    |> List.filter (fun x -> x % 3 = 0 || x % 5 = 0)
    |> List.sum
\end{lstlisting}
\pagebreak

\subsection*{Question 2} 
\begin{origQuestion*}
The Fibonacci sequence (starting with $1$ and $2$) looks something like:
\[
1, 2, 3, 5, 8, 13, 21, 34, 55, 89, \ldots
\]
(For example, $1 + 2 = 3, 2 + 3 = 5, 3 + 5 = 8,$ etc.)

Find the sum of all even-valued fibonacci numbers below $4$ million.
\end{origQuestion*}
\begin{enumerate}
\item We will first test whether the $41$st fibonacci number exceeds four million or not.
\begin{lstlisting}[language=FSharp]
let first40FibNumbers =
    [1 .. 40]
    |> List.scan (fun (x,y) _ -> (y, x + y)) (1,2)
// Result: [(1,2); (2,3); ......; (267914296, 433494437)]
\end{lstlisting}
And so, we only need to consider the first $40$ or $41$ Fibonacci number. (In fact, we do not even need to consider beyond the $32$th number)
\item Sum all even-valued fibonacci numbers below $4$ million.
\begin{lstlisting}[language=FSharp]
let fibSum =
    [1 .. 40]
    |> List.scan (fun (x,y) _ -> (y, x + y)) (1,2)
    |> List.map (fun (x,y) -> x)
    |> List.filter (fun x -> x % 2 = 0)
    |> List.filter (fun x -> x < 4000000)
    |> List.sum
// Result: 4613732
\end{lstlisting}
\end{enumerate}
\subsection*{Question 3} 
\subsubsection*{Exercise (Euler Project Question 3)}

\url{https://projecteuler.net/problem=3}

Please see the next section, where we approach the original Question $3$.

\begin{modQuestion*}
Write a function that takes a list of (positive) integers, and returns the largest prime number in that list.
\end{modQuestion*}

\begin{lstlisting}[language=FSharp]
let FindLargestPrime intList =
    intList
    |> List.filter (IsPrime)
    |> List.max
\end{lstlisting}

\pagebreak
\subsection*{Question 4} 
\url{https://projecteuler.net/problem=4}
\begin{origQuestion*}
A palindromic number reads the same from left-to-right or right-to-left. 

The largest palindromic number made from the product of two 2-digit numbers is $9009 = 91 \times 99$.

Find the largest palindrome made from the product of two 3-digit numbers.
\end{origQuestion*}

The following \texttt{IsPalindrome} function that is already implemented. You do not need to re-implement it.
\begin{lstlisting}[language=FSharp]
let ReverseString (xString: string) =
    new string (xString.ToCharArray() |> Array.rev)

let IsPalindrome xString =
    (ReverseString xString) = xString

let palindromeResult1 = IsPalindrome "ASDF"   
// false

let palindromeResult2 = IsPalindrome "ABCCBA" 
// true
\end{lstlisting}
Find the largest palindrome number which is a product of two 3-digit numbers $a \times b$, where $100 \leq a \leq 999$, and $100 \leq b \leq 999$
\begin{lstlisting}[language=FSharp]
let findProductPalindrome =
    List.allPairs [100 .. 999] [100 .. 999]
    |> List.map (fun (a,b) -> a * b)
    |> List.filter (fun product ->
        product
        |> string
        |> IsPalindrome
    )
\end{lstlisting}
\pagebreak
\subsection*{Question 5} 
\url{https://projecteuler.net/problem=5}

\begin{modQuestion*}
Given a list of integers, find the lowest common multiple (LCM) of all those numbers. (Assume no integer overflow)
\end{modQuestion*}
Remark: We are given the following GCD and LCM functions. We do not need to re-implement them.
\begin{lstlisting}[language=FSharp]
let rec gcd x y =
    if x < 0 || y < 0 then failwith "cannot accept negative numbers"
    if x > y then gcd y x
    else if x = 0 then y
    else gcd (y % x) x

let lcm a b = 
    a * b / (gcd a b)
\end{lstlisting}
Solution:
\begin{lstlisting}[language=FSharp]
let lcmOfList xList =
    xList
    |> List.fold lcm 1

let result11 = lcmOfList [1 .. 10]
// Result: 2520

let result12 = lcmOfList [2;3;4;6;8;12]
// Result: 24
\end{lstlisting}
Remark: This method will FAIL for \texttt{xList = [1 .. 20]} because of integer overflow. Please see the next section on how to handle the original question.
\pagebreak
\subsection*{Question 6} 
\url{https://projecteuler.net/problem=6}
\begin{origQuestion*}
Given a list of integers $x_1, x_2, \ldots, x_n$, write a function that calculates the following:
\[
\left(\sum_{i=1}^n x_i\right)^2 - \left(\sum_{i=1}^n {x_i}^2\right)
\]
\end{origQuestion*}
\begin{lstlisting}[language=FSharp]
let ProjectEulerProblem6 xList =
    let sumOfSquares = 
        xList
        |> List.map (fun x -> x * x)
        |> List.sum

    let sum =
        xList
        |> List.sum

    // return
    (sum * sum) - sumOfSquares
\end{lstlisting}

\begin{lstlisting}[language=FSharp]
let result13 = ProjectEulerProblem6 [1 .. 100]
\end{lstlisting}

\subsection*{Question 7} 

\url{https://projecteuler.net/problem=7}

\begin{origQuestion*}
The list of prime numbers are $2, 3, 5, 7, 11, 13, \ldots$. We can see that the $6$th prime number is $13$. 

What is the $10001$th prime number?
\end{origQuestion*}

We start with a random guess of $500000$:
\begin{enumerate}
\item \textbf{Solution part (a)}: How many prime numbers are there between $2$ and $500000$?
\begin{lstlisting}[language=FSharp]
let numberOfPrimesWithinRange =
    [2 .. 500000]
    |> List.filter (IsPrime)
    |> List.length
\end{lstlisting}
Expected answer: $41538$. 

\item \textbf{Solution part (b)}: What is the $10001$th prime number between $2$ and $500000$?

Because of $0$-based indexing, we use \texttt{(List.item 10000)} to find the $10001$th prime number (which is between $2$ to $500000$).
\begin{lstlisting}[language=FSharp]
let find10001thPrime =
    [2 .. 500000]
    |> List.filter (IsPrime)
    |> List.item 10000
\end{lstlisting}

\end{enumerate}
\subsection*{Question 8} 
\url{https://projecteuler.net/problem=8}

\begin{modQuestion*}
Given a list of digits, find four adjacent digits with the largest product. For example, in the following number:
\[
7316717653133062491922511\mathbf{9674}426574742355349194934
\]
The 4 consecutive digits that gives the largest product is $9 \times 6 \times 7 \times 4 = 9674$

(Notice that this line is the first line in the original question)
\end{modQuestion*}
\begin{lstlisting}[language=FSharp]
let digitList = 
    [7;3;1;6;7;1;7;6;5;3;1;3;3;0;6;2;4;9;1;......]

let result8 =
    digitList
    |> List.windowed 4
    |> List.map (fun x -> x, ListProduct x)
    |> List.maxBy (fun (_,product) -> product)
// val result8 : int list * int = ([9;6;7;4], 1512)
\end{lstlisting}
Please see the next section on how we approach the original question.
\pagebreak
\subsection*{Question 9} 
\url{https://projecteuler.net/problem=9}

\begin{origQuestion*}
Find the only Pythagorean triplet $a, b, c$ that satisfy:
\[
a < b < c, \hspace{1.0cm} a + b + c = 1000, \hspace{1.0cm} a^2 + b^2 = c^2
\]
\end{origQuestion*}
\begin{lstlisting}[language=FSharp]
let FindPythagoreanTriple =
    List.allPairs [1 .. 1000] [1 .. 1000]
    |> List.filter (fun (a,b) ->
        let c = 1000 - a - b
        a * a + b * b = c * c
    )
// Result: [(200, 375); (375, 200)]
\end{lstlisting}
\subsection*{Question 10} 

\subsubsection*{Exercise (Euler Project Question 10)}

\url{https://projecteuler.net/problem=10}

Please see the next section where we work with large numbers (and potential integer overflow)
\begin{modQuestion*}
Given a number $N < 200,000$, find the sum of all prime numbers between $2$ and $N$.
\end{modQuestion*}
\begin{lstlisting}[language=FSharp]
let TotalSumOfPrimeLessThan N =
    [2 .. N]
    |> List.filter (IsPrime)
    |> List.sum
\end{lstlisting}


\pagebreak
\section{Original Project Euler Solutions}
\subsection*{Question 1} 
We did not modify Question 1.
\subsection*{Question 2} 
We did not modify Question 2.
\subsection*{Question 3} 
\url{https://projecteuler.net/problem=3}
\begin{question*}
Given an integer $Z$, write a function that finds the largest prime factor of $Z$. e.g. The prime factors of $13195$ are $5, 7, 13, 29$, and so the largest for $13195$ is $29$.
\end{question*}

\subsubsection*{Problem Analysis}
Remark: Given an integer $Z$, it is possible that the largest prime factor of $Z$ is greater than $\sqrt{Z}$
\begin{itemize}
\item Example: $3 \times 7 = 21$. The largest prime factor is $7 > \sqrt{21} \approx 4.58$.

\item Example: $6 \times 11 = 66$. The largest prime factor is $11 > \sqrt{66} \approx 8.12$.
\end{itemize}
To solve this question, we need some additional mathematical consideration (which is not quite directly related to programming).
\begin{enumerate}
\item Let $S_1 = \{a_1, \ldots, a_n\}$ be all the factors of $Z$ (not necessarily prime factors) between $1$ and $\sqrt{Z}$. This set will always contain at least one element: $a_1 = 1$.
\item Let $S_2 = \left\{\dfrac{Z}{a_1},\ldots, \dfrac{Z}{a_n}\right\}$. These are all the factors of $Z$ between $\sqrt{Z}$ and $Z$. This set will always contain at least one element: $\dfrac{Z}{a_1} = Z$.
\item So, $S_1 \cup S_2 =  \left\{a_1, \ldots, a_n, \dfrac{Z}{a_1},\ldots, \dfrac{Z}{a_n}\right\}$ are all the factor of $Z$ (not necessarily prime factors).
\item Out of our list of candidates $S_1 \cup S_2$, which number is the \underline{largest}, \underline{prime} number?
\end{enumerate}
\pagebreak
\subsubsection*{Working with \texttt{BigInteger}:}
We will need an \texttt{IsPrimeBigInteger} function that helps us check whether a \texttt{BigInteger} is a prime number or not.
\begin{lstlisting}[language=FSharp]
let IsPrimeBigInteger x =
    let squareRoot = x |> double |> sqrt |> BigInteger 
    if x = BigInteger(1) then false
    else if x = BigInteger(2) then true
    else if x % BigInteger(2) = BigInteger(0) then false
    else 
        [BigInteger(3) .. BigInteger(2) .. squareRoot]
        |> List.forall (fun i -> x%i <> BigInteger(0))
\end{lstlisting}
\subsubsection*{Code Solution}
\begin{lstlisting}[language=FSharp]
open System.Numerics

let FindLargestPrimeFactor (Z: BigInteger) =
    let approxSqrt = Z |> double |> sqrt |> BigInteger

    // Find factors of Z between [1 .. sqrt(Z)]
    // Not necessarily prime factors.
    let list1 =
        [BigInteger(1) .. approxSqrt]
        |> List.filter (fun x -> Z % x = BigInteger(0))

    // Find factors of Z between [sqrt(Z) .. Z]
    let list2 =
        list1
        |> List.map (fun a -> Z / a)

    // List.append combines the two lists.
    let combinedList =
        List.append list1 list2
        
    combinedList
    |> List.filter (IsPrimeBigInteger)
    |> List.max
\end{lstlisting}
\subsection*{Question 4} 
We did not modify Question 4.

\pagebreak
\subsection*{Question 5} 
\url{https://projecteuler.net/problem=5}
\begin{origQuestion*}
Find the least common multiple (LCM) of $1$ to $20$.
\end{origQuestion*}
\begin{lstlisting}[language=FSharp]
open System.Numerics

let rec GCD x y =
    let zero = BigInteger(0)
    if x < zero || y < zero then failwith "CANNOT ACCEPT NEGATIVE NUMBERS"
    if x > y then GCD y x
    else if x = zero then y
    else GCD (y % x) x

let LCM x y = 
    x * y / (GCD x y)

let result =
    [1 .. 20]
    |> List.map (BigInteger)
    |> List.reduce LCM
\end{lstlisting}
\subsection*{Question 6} 
We did not modify Question 6.
\subsection*{Question 7} 
We did not modify Question 7.
\subsection*{Question 8} 
\url{https://projecteuler.net/problem=8}
\begin{origQuestion*}
In the webpage, a $1000-digit$ number is provided.

The four adjacent digits in the $1000$-digit number that have the greatest product are $9 \times 9 \times 8 \times 9 = 5832$.
\begin{center}
\texttt{731671765313......31998900.......2963450}
\end{center}
Find the thirteen adjacent digits in the $1000$-digit number that have the greatest product. 
\end{origQuestion*}
\begin{lstlisting}[language=FSharp]
open System.Numerics

let bigIntegerString =
    "7316717653133062491...........20752963450"
// Please actually fill in the actual digits in the actual F# file!

let result = 
    bigIntegerString
    |> Seq.map (string)
    |> Seq.map (int)
    |> Seq.map (BigInteger)
    |> Seq.windowed 13
    |> Seq.map (Seq.reduce (*))
    |> Seq.max
\end{lstlisting}
Remark: You can consider \texttt{bigIntegerString} as one long string, or a character \texttt{char} array.

So, the first \texttt{Seq.map (string)} converts an array of \texttt{char} to an array of \texttt{string} (where each string has only one letter/digit).
\subsection*{Question 9} 
We did not modify Question 9.
\subsection*{Question 10} 
\url{https://projecteuler.net/problem=10}
\begin{origQuestion*}
The sum of the primes below $10$ is $2 + 3 + 5 + 7 = 17$

Find the sum of all the primes below two million $(2,000,000)$.
\end{origQuestion*}

\begin{lstlisting}[language=FSharp]
open System.Numerics

let Version2_TotalSumOfPrimeLessThan N =
    [2 .. N]
    |> List.filter (IsPrime)
    |> List.map (BigInteger)
    |> List.sum

// Remark: The code below can take 10 seconds, as this is not the most optimal algorithm.
let result17 = Version2_TotalSumOfPrimeLessThan 2000000
// Result: 142913828922
\end{lstlisting}

\end{document}