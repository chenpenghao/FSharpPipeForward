\documentclass[12pt]{article}

\usepackage{geometry}
\geometry{
	a4paper,
 	left=26mm,
 	right=26mm,
 	top=33mm,
 	bottom=38mm
}
\usepackage{color}
\definecolor{bluekeywords}{rgb}{0.13,0.13,1}
\definecolor{greencomments}{rgb}{0,0.5,0}
\definecolor{redstrings}{rgb}{0.9,0,0}

\usepackage{listings}
\lstdefinelanguage{FSharp}%
{morekeywords={let, new, match, with, rec, open, module, namespace, type, of, member, % 
and, for, while, true, false, in, do, begin, end, fun, function, return, yield, try, %
mutable, if, then, else, cloud, async, static, use, abstract, interface, inherit, finally },
  otherkeywords={ let!, return!, do!, yield!, use!, var, from, select, where, order, by },
  keywordstyle=\color{bluekeywords},
  sensitive=true,
  basicstyle=\ttfamily,
	breaklines=true,
  xleftmargin=\parindent,
  aboveskip=\bigskipamount,
	tabsize=4,
  morecomment=[l][\color{greencomments}]{///},
  morecomment=[l][\color{greencomments}]{//},
  morecomment=[s][\color{greencomments}]{{(*}{*)}},
  morestring=[b]",
  showstringspaces=false,
  literate={`}{\`}1,
  stringstyle=\color{redstrings},
}

\begin{document}


\begin{center}

{\large F\# Tutorial\\} \vspace{2mm}
\textbf{\LARGE Pipe-Forward Operator}\\
\vspace{1.5mm}
{\Large\emph{\today}}

\end{center}


\section{Syntax, variables, functions}

\subsection{Key concepts: } 

\begin{enumerate}
\item Having a good text editor helps you code much easier.
\item 
\begin{enumerate}
\item Once defined, a variable in F\# cannot change value (unless "mutable" is used)
\item If you need an updated value, create a new one.
\end{enumerate}
\item Different datatypes (e.g. integer and decimal-numbers) do not combine easily.
\item Defining and using functions in F\# is slightly different from math notation/ other languages.
\begin{enumerate}
\item F\# automatically detects the type of the variables (e.g. integer, double, etc.) for a function.
\item The variable types for a function will be enforced.
\end{enumerate}
\end{enumerate}

\subsection{Introduction: } 
\subsubsection{Comments}
You can use double-slash \texttt{//}, triple-slash \texttt{///}, or star-bracket \texttt{(* ...... *)} to make comments.

\begin{lstlisting}[language=FSharp]
// These words are ignored.
/// These words are ignored.
(* These words are ignored. *)
let x = 1
let y = x + 5
\end{lstlisting}

\vfill

\pagebreak

\subsubsection{Intellisense}
If you are using Visual Studio or Visual Studio Code, you can put your mouse on top of the variable name \texttt{x} or \texttt{y}, and see that it is an \texttt{int} or integer.

This feature will help you identify what is each variable/function, and make coding easier for you.
\begin{center}
SHOW PICTURE OF INTELLISENSE HERE.
\end{center}

\subsubsection{Common data types and printing}

Some of the common types in F\# are:
\begin{center}
\begin{tabular}{|c|c|c|}
\hline Keyword & Description & Print in output:
\\ \hline \texttt{int} & Integer & \texttt{\%i}
\\ \hline \texttt{double} or \texttt{float} & Decimal numbers & \texttt{\%f}
\\ \hline \texttt{string} & Words/Sentences & \texttt{\%s}
\\ \hline \texttt{bool} & True/False & \texttt{\%b}
\\ \hline - & Other objects & \texttt{\%A} or \texttt{\%O}
\\ \hline
\end{tabular}
\end{center}

\begin{lstlisting}[language=FSharp]
let name = "John"
let age = 21
let height = 170.5

printfn "My name is: %s" name
// Output:
// My name is: John

printfn "Name: %s. Age: %i. Height: %f." name age height
// Output:
// Name: John. Age: 21. Height: 170.500000

printfn "His height is: %.2f" height
// Output:
// His height is: 170.50
//// Show only two decimal.
\end{lstlisting}
For example, in the second example, inside the string-format, there are \texttt{\%s, \%i, \%f}. And so, we expect a string, integer, and decimal (in that order) after the string-format specification in order to completely print the result to the output console.

\vfill

\pagebreak

\subsubsection{Immutability}

\end{document}